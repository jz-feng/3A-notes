\documentclass[11pt, oneside]{article}
\usepackage[margin=0.75in]{geometry}
\geometry{letterpaper}
\usepackage{amssymb}
\usepackage[fleqn]{amsmath}
\usepackage[sharp]{easylist}
\usepackage{relsize}
\usepackage{graphicx}
\usepackage{enumitem}

\pagenumbering{gobble}              % No page numbering
\setlength{\parindent}{0em}         % No paragraph indenting
\setlength{\parskip}{0.5em}         % Paragraph spacing

\newcommand*{\begineasylist}{\begin{easylist}[itemize]\ListProperties(Style*=$\bullet$\quad, Style2*=\tiny$\blacksquare$\quad, Style3*=$\circ$\quad, Style4*=$\diamond$\quad, FinalSpace=1em, Space=-0.2em, Space*=-0.2em)}

\newcommand*{\begineasylistnumbered}{\begin{easylist}[enumerate]\ListProperties(Numbers=a, Space=0em, Space*=0em)}

\newcommand*{\un}[1]{\underline{\smash{#1}}}        % custom underline macro

\newenvironment{itemized}{\begin{itemize}[noitemsep, topsep=0pt, leftmargin=*]}{\end{itemize}}  % cusom itemize environment

\begin{document}

\section*{CS 349 Midterm Review}

\subsection*{Background \& History}
\begineasylist

# \textbf{User interface}: 
## The place where a person \un{expresses intention} to an artifact, and the artifact \un{presents feedback} to the person
## The way people (mental model) and technology (system model) interact

## Represented as MVC: \\
% \begin{figure}[h]
\includegraphics[width=0.7\textwidth]{res/intro_mvc.png}
% \end{figure}

# \textbf{Interface}: external presentation (visual, physical, auditory) to the user
## e.g. controls
# \textbf{Interaction}: actions invoked by user and corresponding responses (behaviour)
## e.g. action and dialog

# Batch interfaces (1945-1965)
## Sets of instructions fed via punch cards; only used by highly trained individuals

# Conversationalist interface (1965-1985+)
## Text-based feedback and input; I/O is in system language, not task language

# Graphic user interface (1984+)
## \textbf{WIMP interface}: windows, icons, menus \& pointer
## Benefits of GUI:
### Keeps the user in control
### Emphasize recognition (discovery of options) over recall (memorizing commands)
### Uses metaphor; makes interaction language closer to user's language

# Notable people:
## Vannevar Bush -- conceptualized the memex, a desk with integrated display, input, and data storage
## Ivan Sutherland -- created the Sketchpad, an early graphical interface with a light pen and direct manipulation
## Douglas Engelbart -- invented the mouse, introduced copy/paste
## Alan Kay -- worked on the Xerox Star, first commercial computer with GUI

\end{easylist}

% ============================

\newpage
\subsection*{Windowing Systems \& X11}
\begineasylist

# \textbf{Windowing system}: provides input, output, and window management capabilities to the OS

# \textbf{X Windows (X11)}:
## Standard windowing system for Unix-based systems

# X11 architecture
## \un{X Client} handles all application logic
## \un{X Server} handles all user input \& display output
## There may be many clients -- each client is an application; server draws all clients onto one screen and reads all input \\
\includegraphics[width=0.7\textwidth]{res/mvc_x11.png}

# Structure of an X program (application is run on the X client):
## Perform client initialization
## Connect to X server (e.g. \texttt{XOpenDisplay()}, \texttt{XCreateWindow()})
## Perform X related initialization (e.g. create graphic contexts with \texttt{XCreateGC()}; put window on the screen with \texttt{XMapRaised()})
## Event loop
### Get next event from server (e.g. \texttt{XNextEvent()})
### Handle event (e.g. \texttt{XLookupKeysym()})
### Send draw request to server (e.g. flush output buffer with \texttt{XFlush()})
## Close down connection to X server (e.g. \texttt{XCloseDisplay()})
## Perform client cleanup

# X11 is a \textbf{base windowing system}: \\
\includegraphics[width=0.4\textwidth]{res/bws.png}
## A standard/protocol for creating windows, low-level graphical output, and user input
## Does \un{not} specify the style of each application's UI
## Provides each application with a window and manages its access
## Each application (only) owns a \un{canvas}; shielded from details such as visibility, other windows

## Some \textbf{design goals of X11/BWS}:
### Display- \& device- independent
### Supports multiple overlapping \& resizable windows
### A display may have multiple screens (monitors) and a window may span multiple screens
### High-performance, high-quality text, 2D graphic \& imaging

# \textbf{Window manager}:
## Provides interactive components (e.g. menus, close button, resizing)
## The WM owns each application's window itself (while application owns the canvas)
### i.e. application developers usually cannot change the window style
## Separation of the WM from the BWS enables many alternative ``look and feels'' \\
\includegraphics[width=0.5\textwidth]{res/wm.png}

# \textbf{Drawing}
## Three conceptual drawing models:
### Pixel (e.g. images)
### Stroke (e.g. lines, outlines of shapes)
### Region (e.g. text, filled shapes)

## X11 uses \un{graphics contexts} to store drawing options/parameters -- stored on X server

## \textbf{Clipping}: exposing only a particular region (specified by a \un{mask}) of an underlying image
### \texttt{XSetClipMask()}, \texttt{XSetClipRectangles()}
### Only exposed area is repainted -- more efficient

## \textbf{Painter's Algorithm}: draw shapes in layers from back to front to create composite shapes
### \texttt{Displayable} class with abstract \texttt{paint()} method; implement \texttt{paint()} in each subclass
### Draw list of \texttt{Displayables} from back to front, clear screen on every repaint

# \textbf{Events \& animation}
## Objective: need to map input from real-word devices to actions within a system
## \textbf{Event-driven programming}: flow of program is determined by \un{events} such as user input (key press, mouse click, input focus change) or messages from other programs/threads
## Events are pushed into an \un{event queue} by the BWS (i.e. \un{event capture})
## Implementation in X11:
### Use \texttt{XSelectInput()} and event masks (e.g. \texttt{KeyPressMask} etc.) to register/subscribe to types of events
#### Filters out unneeded events
### Use \texttt{XNextEvent()} to dequeue the next event; \emph{may block if no events}
#### Use \texttt{XPending()} to check for \# of events before dequeueing
### Should dequeue \emph{all events} before repainting to avoid input lag
### Should subtract time spent in event loop from \texttt{sleep()} to maintain consistent FPS
### Should draw all images to a \emph{buffer} (\texttt{XCreatePixmap()}), then copy the buffer onto the screen in one go (\texttt{XCopyArea()}) (aka. \un{double buffering})
#### Avoids displaying an intermediate image (i.e. flickering)

\end{easylist}


% ============================

\newpage
\subsection*{Widgets \& Events}
\begineasylist

# \textbf{Widgets}: parts of an interface that have their own behaviour
## Control their own appearance; recieve and handle their own events
## Widgets \un{toolkit} defines a set of GUI components
## Design goals: 
### \un{Complete} -- covers wide range of functionality
### \un{Consistent} -- look-and-feel across components
### \un{Customizable} -- developers can extend functionality

## Consistent behaviour of components helps users anticipate how the interface will react, and promotes easier \emph{discoverability} of features

\hspace{-2em}
\begin{tabular}{|l|l|}
\hline
\begin{minipage}[t]{0.45\textwidth}
\textbf{Heavyweight widgets}:
    \begin{itemized}
    \item Wrappers around OS's native GUI \& windowing system 
    \item e.g. Java AWT
    \end{itemized}
    \vspace*{0.5em}
\end{minipage}
&
\begin{minipage}[t]{0.45\textwidth}
\textbf{Lightweight widgets}:
    \begin{itemized}
    \item OS provides top-level window in which widgets are drawn
    \item Toolkit is responsible to passing events to widgets
    \end{itemized}
    \vspace*{0.5em}
\end{minipage}  \\
\hline
\begin{minipage}[t]{0.45\textwidth}
Advantages:
    \begin{itemized}
    \item Events passed directly to OS/BWS
    \item Preserves the OS look-and-feel
    \end{itemized}
    \vspace*{0.5em}
\end{minipage}
&
\begin{minipage}[t]{0.45\textwidth}
Advantages:
    \begin{itemized}
    \item Consistent look-and-feel across platforms
    \item Consistent widget set across platforms
    \item Allows for highly optimized widgets
    \end{itemized}
    \vspace*{0.5em}
\end{minipage}  \\
\hline
\begin{minipage}[t]{0.45\textwidth}
Disadvantages:
    \begin{itemized}
    \item OS-specific programming
    \item Small set of common widgets across different platforms
    \end{itemized}
    \vspace*{0.5em}
\end{minipage}
&
\begin{minipage}[t]{0.45\textwidth}
Disadvantages:
    \begin{itemized}
        \item May appear ``non-native''
    \end{itemized}
    \vspace*{0.5em}
\end{minipage}  \\
\hline
\end{tabular} \\

# Widgets as logical input devices
## Characteristics:
### \un{Model} manipulated by the widget (e.g. number, text)
### \un{Events} generated by the widget (e.g. changed)
### \un{Properties} (behaviour and appearance) of the widget (e.g. colour, size, allowed values)
### e.g. radio button: model = \emph{Boolean}; events = \emph{changed}; properties = \emph{size, colour etc.} \\
\includegraphics[width=0.5\textwidth]{res/mvc_widget.png}

## Model is abstracted into an interface/abstract class for more code reuse and customizability
### Interface may provide many accessors, mutators \& event-firing functions to be implemented by the custom widgets, allowing for easy manipulation of custom data

## Essential geometry is computed by the view; controller interacts with it

# Event dispatch $\rightarrow$ event handling $\rightarrow$ notifying view \& windowing system (MVC)

# \textbf{Event dispatch}: dequeueing events from event queue and pushing to appropriate applications

# \textbf{Interactor tree} -- hierarchy of containers and their nested widgets

# \textbf{Positional dispatch} -- input sent to widget under mouse cursor location
## \un{Bottom-up dispatch}:
### Event is routed to leaf (lowest) widget in interactor tree
### Widget can process the event or pass to its parent
### e.g. widget belongs in a group/container -- may be better for container to handle the event
### Advantage: event does not have to traverse through entire tree to arrive at widget
## \un{Top-down dispatch}:
### Event is routed to highest-level node that contains mouse cursor
### Widget can process the event or pass to child component
### Advantages:
#### Parent widget can enforce policies (e.g. make children view-only)
#### Easy event logging (as it traverses down through the tree)
## Pure positional dispatch can be problematic
### e.g. mouse-down inside a button, mouse-up outside; dragging scrollbar but mouse moves out of scrollbar

# \textbf{Focus dispatch} -- events dispatched to widget that has keyboard/mouse focus
## At most one widget each can be in keyboard \& mouse focus at a given time
## Focus dispatch also needs positional dispatch to change focus (i.e. mouse click)
## Accelerator keys (i.e. keyboard shortcuts) can bypass focus dispatch -- they're handled before widget receives events

\hspace{-2em}
\begin{tabular}{|l|l|}
\hline
\begin{minipage}[t]{0.45\textwidth}
\textbf{Heavyweight toolkits}:
    \begin{itemized}
    \item BWS has visibility to all widgets
    \item Can use top-down or bottom-up dispatch
    \end{itemized}
    \vspace*{0.5em}
\end{minipage}
&
\begin{minipage}[t]{0.45\textwidth}
\textbf{Lightweight toolkits}:
    \begin{itemized}
    \item BWS only has visibility to application window
    \item Toolkit then dispatches event to widget
    \item Can only use top-down dispatch
    \end{itemized}
    \vspace*{0.5em}
\end{minipage}  \\
\hline
\end{tabular} \\

# \textbf{Event handling}: interpreting events in widget's application code
## Design goals of event-code binding:
### Easy to understand
### Easy to implement
### Easy to debug
### Good performance

## Event loop \& switch statement (X11):
### All events are consumed in one event loop
### Switch statement selects the appropriate code for each event
### Downsides: switch statement needs to encompass every type of event (too many!)

## Inheritance binding (Java, OS X):
### Events are dispatched to base widget class with predefined event handling methods
### Child widget overrides methods with custom behaviour
### Downsides:
#### Event handling code in application logic (child widget) -- no separation of concerns
#### Difficult to add new events

## Listener binding (Java):
### \un{Interface binding} -- widget class implements event listener interfaces

\hspace{2em} \texttt{public class A implements Listener \{ // implement all methods \}}

### \un{Object binding} -- widget class holds listener objects (implement listener interface as a nested class)
#### Event handling \& application code are decoupled

\texttt{this.addListener(new Listener() \{ // implement all methods \});}

### \un{Adapter pattern} -- widget class holds adapter objects (class with boilerplate implementations)
#### Custom adapter only needs to extend methods that are used

\texttt{this.addListener(new ListenerAdapter() \{ // override some methods \});}

## Delegate binding (.NET):
### Delegates ``point'' to a method (or methods); invoking delegate calls all associated methods

\hspace{2em} \texttt{delegate = object.Method1; delegate += object.Method2; delegate(args);} \\

\end{easylist}


% ============================

\newpage
\subsection*{Layouts}
\begineasylist

# \textbf{Dynamic layout} -- dynamically adjusts screen composition; provides spatial layout for widgets in a container
## Handles container resize by adjusting location, size, visibility or look-and-feel of widgets
# \textbf{Adaptive/responsive layout} -- may go beyond spatial layout in order to adapt to different device sizes
# Widgets may define constraints for size (e.g. min, preferred, max), position (e.g. anchors)
# \un{Layout managers} provide algorithms to size \& position widgets

# \un{Composite pattern} -- group/container of widgets and individual widgets are treated uniformly
## Widgets are organized in a tree hierarchy

# \un{Strategy pattern} -- abstract out the algorithm so that it can be changed at run-time
## Layout manager can employ different layout strategies

# Types of layouts:
## \un{Fixed} -- widgets have fixed size \& position
### e.g. set \texttt{LayoutManager} to null

## \un{Intrinsic size} -- parent widget's size depends on contained widgets
### Bottom-up algorithm -- query each child widget for preferred size, then set size for parent
### e.g. \texttt{BoxLayout}, \texttt{FlowLayout}

## \un{Variable intrinsic size} -- widget size depends on both parent and contained widgets
### Bottom-up \& top-down algorithm
### e.g. \texttt{GridBagLayout}, \texttt{BorderLayout}

## \un{Struts and Springs} -- layout specified by contraints and anchors
### Strut widgets are fixed in size; spring/glue widgets stretch to fill space
### e.g. \texttt{SpringLayout} \\
\includegraphics[width=0.6\textwidth]{res/struts_and_springs.png}

\end{easylist}


% ============================

\newpage
\subsection*{Graphics \& Transformations}
\begineasylist

# \textbf{2D graphics}:
## Shape model -- data needed to draw a shape (array of points, colour, location etc.)
## Rendering -- using the properties to create an image to be displayed on the screen \\
\includegraphics[width=0.4\textwidth]{res/graphics.png}

# \textbf{NOTE}: origin is located at the \un{top-left} when discussing graphics \& transformations

# Selection paradigms:
## \un{Click selection (for lines)} -- find closest line segment to mouse position
### Check distance from mouse to each line segment using vector projection
### Count as ``selection'' for distance under a certain threshold
## \un{Click selection (for closed shapes)} -- check if mouse position is within shape
### For complex polygons, draw a ray extending from the point \& count the \# of intersections with the polygon's boundary
### If odd \# of intersections, the point is within the polygon; if even \#, it is not

# \textbf{Affine transformations}:
## \textbf{Translation}: add scalar to $x$ and/or $y$ component
## \textbf{Scaling}: multiply $x$ and/or $y$ components by scalars
## \textbf{Rotation} (about the origin): $x' = x\cos(\Theta) - y\sin(\Theta)$, $y' = x\sin(\Theta) + y\cos(\Theta)$

## Order of operations: scale $\rightarrow$ rotate $\rightarrow$ translate
### $x' = s_x(x\cos(\Theta) - y\sin(\Theta)) + t_x$
### $y' = s_y(x\sin(\Theta) + y\cos(\Theta)) + t_y$
### Since scaling \& rotation are about the origin, should \un{translate to origin first}, and translate back after scaling/rotation
## Translation can't be done using $2 \times 2$ matrix -- use \un{homogeneous coordinates}
### $[x, y, w]$ represents a point at $[x/w, y/w]$; e.g. $[1, 2, 1] = [2, 4, 2]$
## \textbf{Affine transformation matrix}: calculates all transformations using $3 \times 3$ matrix
\[
    \begin{bmatrix} x' \\ y' \\ 1 \end{bmatrix}
    =
    \begin{bmatrix}
        1 & 0 & t_x \\
        0 & 1 & t_y \\
        0 & 0 & 1
    \end{bmatrix}
    \begin{bmatrix}
        \cos(\Theta) & - \sin(\Theta) & 0 \\
        \sin(\Theta) & \cos(\Theta) & 0 \\
        0 & 0 & 1
    \end{bmatrix}
    \begin{bmatrix}
        s_x & 0 & 0 \\
        0 & s_y & 0 \\
        0 & 0 & 1
    \end{bmatrix}
    \begin{bmatrix} x \\ y \\ 1 \end{bmatrix}
\]
### Transformations are applied right to left $\leftarrow$

# \textbf{Scene graph} -- each component has a transformation matrix \& draws its child components relative to itself
## The interactor tree is a type of scene graph
## Each component has a transformation matrix (describes its location relative to parent)
### Paints itself, then
### Combine its matrix with child component's matrix, and tells child to paint itself using combined matrix

# Benefits of geometric manipulation:
## Allows reuse of objects (create multiple instances via transformations)
## Allows specification of object in its own coordinate system (e.g. relative to parent)
## Simplifies repositioning of object after change (e.g. moving an object in animation)

# Coordinates given by events need to be transformed as they traverse the interactor tree
## e.g. for \un{inside tests/hit detection}, mouse event coordinates must be transformed into a model's local coordinates 
## Transforming \un{mouse $\rightarrow$ model} coordinates
### Only one transformation (of the mouse event) -- take the inverse of model's affine matrix
## Transforming \un{model $\rightarrow$ mouse} coordinates
### Many transformations (of all objects in the scene) in order to find which one the mouse is inside of



% ============================

\end{easylist}

\newpage
\subsection*{Model-View-Controller}
\begineasylist

# \textbf{MVC} -- multiple views \emph{loosely coupled} with the underlying data model
## Developed for Smalltalk-80 by Trygve Reenskaug
## Tight coupling of data \& presentation prevents easy modification and extension
## \un{Separation of concerns} enables:
### Alternate forms of interaction/presentation with the same data
### Multiple, simultaneous views of data
### Easy testing of data manipulations that are independent of the UI
## \un{View} \& \un{controller} can access the \un{model} through its interface; model only knows about the view
### Controller $\rightarrow$ (notifies) $\rightarrow$ Model
### View $\rightarrow$ (queries) $\rightarrow$ Model
### Model $\rightarrow$ (updates) $\rightarrow$ View
### Controller \& view are tightly coupled in practice 
#### Controller is just part of the view class that calls the model's interface based on input \\
\includegraphics[width=0.4\textwidth]{res/mvc_practice.png}

# MVC is an instance of the \textbf{observer pattern}
## Allows objects to communicate without knowing each others' specific types
## In Java, the view implements \texttt{Observer} (like \texttt{IView}); model extends \texttt{Observable} \\
\includegraphics[width=0.6\textwidth]{res/mvc_observer.png}


\end{easylist}


% ============================

\newpage
\subsection*{Input}
\begineasylist

# Computer input can be classified by sensing method (e.g. mechanical, motion, contact), continuous vs. discrete, degrees of freedom
# Devices are mostly focused on text \& positional input
# \textbf{Text input}
## QWERTY has many \emph{perceived} problems:
### Many common combinations require inefficient finger movements
### Most typing is done with left hand
### Most typing is \emph{not} done on the home row
## Dvorak attempts to address these problems, but actual difference in speed is discernible
## Portability (smaller, lower-profile keys) of keyboards also interfere with typing performance
## Soft/virtual keyboards lack haptic feedback, but improves aethestics -- good for when the amount of input is limited

# \textbf{Positional input}
## \un{Isometric} (force) vs. \un{isotonic} (displacement) sensing
### Device senses displacement (mouse) or force (joystick)
## \un{Position} vs. \un{rate} control
### Change in input device maps to change in position (mouse) or speed (joystick)
### Usually, isometric $\rightarrow$ rate, isotonic $\rightarrow$ position
## \un{Absolute} vs. \un{relative} position
### 1:1 mapping between input \& output position (touchscreen) or non-1:1 mapping (mouse)
## \un{Direct} vs. \un{indirect} contact
### Input takes place on the same surface as output (touchscreen) or on a different surface (mouse)
## Dimensions sensed -- 1 (dial) vs. 2 (mouse) vs. 3 (Wiimote)


\end{easylist}

\end{document}
